\chapter{Objetivos}
\label{chap:objetivos}

En este capítulo se describen los objetivos que se marcan para la consecución del
Proyecto Fin de Carrera, así como los medios técnicos que se emplean para su óptima
consecución.

\section{Objetivo principal} \label{sec:objetivo-principal}

El objetivo principal de este PFC es:

\emph{Diseñar y desarrollar un sistema de soporte para la migración de sistemas heredados en explotación
desarrollados en PowerBuilder y escritos en PowerScript hacia platformas actuales que están actualmente
en explotación, a una tecnología de código abierto actual.}

De esta forma se posibilitará el aprovechamiento 
del conocimiento acumulado en las reglas de negocio que durante la fase de explotación se han implementado y 
su utilización como base en una futura aplicación informática. 

\subsection{Objetivos parciales}\label{sec:objetivos-parciales}
Para lograr este objetivo principal se plantean los siguientes objetivos parciales:
\begin{description}

\item[Obj.1] Análisis de la arquitectura de las aplicaciones PowerBuilder, del lenguaje de desarrollo con el que se escriben 
y estudio en detalle del código fuente disponible que generan. Con ello se pretende obtener un conocimiento pormenorizado de cada uno de los
objetos, funciones, tipos de datos y estructuras del lenguaje. 

\item[Obj.2] Generación de una gramática que permita reconocer el lenguaje PowerScript mediante el uso del analizador
sintáctico ANTLR. Este gramática será liberado para su posterior uso por parte de la comunidad de desarrolladores y su 
inclusión dentro de la colección de gramáticas de la herramienta de análisis en el repositorio Github de la comunidad.

\item[Obj.3] Diseño y Desarrollo de un módulo para la generación de modelos basados en KDM (Knowledge-Discovery Metamodel)\cite{KDM}
a partir de la información extraída del analizador PowerBuilder.


\end{description}

\subsection{Objetivos docentes} \label{sec:objetivos-docentes}

Para la consecución del objetivo principal será necesario llevar a cabo los siguientes objetivos docentes:
\begin{itemize}
\item Estudiar las técnicas de reingeniería e ingeniería inversa.
\item Análisis de la sintaxis de los lenguajes de programación.
\item Aprender a crear analizadores sintácticos mediante el diseño de gramáticas.
\item Perfeccionar los conocimientos y mejorar la experiencia actual en técnicas de testing y de los procesos de desarrollo basados en tests.	
\end{itemize}

\subsection{Recursos tecnológicos} \label{sec:recursos-tecnológicos}

\begin{itemize}
 \item Ordenador portátil Dell Inspiron 14z con procesador CoreI5 y 6Gb de memoria RAM. El sistema operativo sobre el que se implementa es
 Linux Mint 17
 \item Como herramienta de diseño para los distintos diagramas se han utilizado tanto la herramienta DIA como la aplicación online Cacoo.
 \item Para la redacción de la documentación se ha empleado el lenguaje \LaTeX{} y la herramienta gráfica Kile v2.13 de KDE en combinación 
 con la clase \textbf{arco-pfc}
 
\end{itemize}
