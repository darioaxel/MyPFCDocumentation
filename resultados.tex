\chapter{Resultados}
\label{chap:Resultados}
En este capítulo se presentan los resultados de las iteraciones descritas en la sección anterior.
\section{Iteración 0: Planificación de las iteraciones}
\begin{figure}[!h]
\begin{center}
\includegraphics[width=1.1\textwidth]{/ProjectMainDiagram.png}
\caption{Base arquitectural del sistema}
\label{fig:base-architecture}
\end{center}
\end{figure}
\section{Iteracción 1: Desarrollo de la gramática ANTLRv4 de PowerBuilder}

En la primera iteración del proceso de desarrollo se cumplimentaran todas las fases de un ciclo de la metodología, además de 
incorporar varios ciclos de desarrollo basado en TDD. El objetivo principal de esta iteración tal y como se definió en 
la fase 0 del proyecto será: \emph{Generar una gramática que permita procesar el lenguaje Powerscript}.

\subsection{Planificación y requerimientos}

Para abordar esta parte de la iteración, inicialmente se realiza una busqueda de gramáticas que pudiesen servir de base 
en la creación de la que se plantea como objetivo. Después de investigar acerca de las gramáticas ANTLRv4 y tras descartar
las que se puediera encontrar una propia del lenguaje de programación a estudio en el repositorio github del proyecto\cite{}
se pasa a realizar un análisis pormenorizado de una colección de ficheros de código extraidos del sofware heredado. 
Como apoyo se utilizará tanto la guía de referencia del lenguaje \cite{PowerBuilder} como las estructuras de otras gramáticas del proyecto
ANTLRv4 que puedan servir de orientación.

\subsection{Análisis y diseño}

Una vez analizados numerosos ficheros de código fuente se establecen dos conjuntos de artefactos a tratar:

\begin{itemize}
 \item Estructuras comunes con otros lenguajes de programación conocidos.
 \item Estructuras propias de Powerscript.
\end{itemize}

A continuación se abordan ambos grupos de artefactos y en base a ellos se identifican los test mínimos que una vez resueltos
en fases siguientes de la iteracción permitan alcanzar el objetivo propuesto.

\subsubsection{Estructuras comunes} \label{subsec:EstructurasComunes}

Se analizarán ahora todas aquellas estructuras que se puedan relacionar con las mas comunes de otros lenguajes de programación conocidos.

\paragraph{Palabras reservadas del lenguaje, tipos de datos y comentarios}
Tanto las palabras reservadas como los tipos de datos, pueden encontrarse facilmente en los primeros capítulos de la guía de desarrollo \cite{PowerBuilder}.
Además los comentarios no difieren de los que se utilizan en los lenguajes de programación mas usuales. Por tanto, es sencillo establecer un primer test
que además nos servirá para comprobar el entorno de desarrollo establecido en la iteracción anterior. 

El primer test, denominado «TestComments.txt» se crea y adjunta al proyecto en la carpeta: \verb|\resources\basics|

\paragraph{Declaración de variables y constantes}
La forma en la que se declaran las variables y las constantes, salvo por los tipos propios del lenguaje, también es idéntica a la que se utiliza en otros 
lenguajes de programación. 

En este punto se ha de resaltar una característica propia del lenguaje a tener en cuenta y que va a condicionar todo el desarrollo de la gramática: 

\textbf{No existen identificadores terminales de las sentencias del lenguaje Powerscript}.

\begin{listing}[
  float=ht,
  language = Powerscript,
  caption  = {Definición de variables},
  label    = code:variables]
type variables
boolean ib_Painting
integer width = 251, height = 72
fontcharset fontcharset = ansi
end variables
\end{listing}

Como se puede comprobar en el listado \ref{code:variables}, la única forma de discernir cuándo termina una sentencia y comienza otra es mediante el objeto 
\verb|\n| que marca el fin de la línea de texto. 

Para validar esta parte de la gramática se establecen una serie de test que se adjuntan a la carpeta: \verb|\resources\literals| y \verb|\resources\members\constants|

\paragraph{Bucles}

Dentro de Powerscript encontramos los siguientes bucles de flujo:
\begin{itemize}
 \item IF-THEN-ELSEIF-THEN 
 \item FOR 
 \item DO-WHILE
 \item TRY-CATCH
 \item CHOOSE
\end{itemize}
Para validar esta parte de la gramática se establecen una serie de test que se adjuntan a la carpeta \verb|\resources\statements| , con test definidos para los distintos bucles
que se describen.

\paragraph{Sentencias SQL}
Dentro del código de un objeto PowerBuilder existe la capacidad de realizar directamente sentencias SQL y que son lanzadas a una conexión previamente establecida a una base de datos.
Esta parte de la gramática es importante puesto que uno de los requisitos establecidos por la empresa Savia, es posibilitar la recuperación de las llamadas a base de datos desperdigadas
por el sistema, que hacen de la tarea de modificación de cualquier campo de una base de datos un proceso muy tedioso.

\begin{itemize}
 \item COMMIT
 \item CONNECT
 \item DISCONNECT
 \item ROLLBACK
 \item SELECT
 \item UPDATE
 \item INSERT
 \item DELETE	
 \item OPENCURSOR
 \item CLOSECURSOR
 \item DECLARECURSOR
 \item FETCHCURSOR
\end{itemize}
 

\paragraph{Funciones}



\paragraph{Expresiones}

\subsubsection{Estructuras propias}

En esta sección se analizarán las estructuras propias del 
\paragraph{Estructura «forward»}
\paragraph{Estructura «type»}
\paragraph{Estructura «variables»}
\paragraph{Estructura «framework»}
\paragraph{Estructuras «event» y «on»}
\paragraph{Estructuras «prototipos de funciones»}
\paragraph{Comandos propios}

\subsection{Desarrollo, test y validación}

Durante el proceso de desarrollo se realizan 5 iteraciones dirigidas por los bloques de test establecidos en la fase anterior de la iteración:

Como muestra de ello se adjunta el listado\ref{fig:common-structures} resultante del primer bloque de testing que incluye las «Estructuras comunes»\ref{subsec:EstructurasComunes}

\begin{figure}[!h]
\begin{center}
\includegraphics[width=0.8\textwidth]{/Testing01.png}
\caption{Resultados del testing de estructuras comunes}
\label{fig:common-structures}
\end{center}
\end{figure}

\begin{figure}[!h]
\begin{center}
\includegraphics[width=0.8\textwidth]{/Testing02.png}
\caption{Resultados del testing de la gramática completa}
\label{fig:full-grammar-test}
\end{center}
\end{figure}

\section{Iteracción 2: Sistema de validación del software heredado}
\section{Iteracción 3: Generación del InventoryModel}
\section{Iteracción 4: Tranformación de tipos internos}
\section{Iteracción 5: Transformación de miembros internos y métodos}
\section{Iteracción 6: Transformación de sentencias}